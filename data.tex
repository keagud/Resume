
\begin{document}
    \resumetitle{Keaton}{Guderian} {
      keagud@protonmail.com | (206)-321-2181
    }

    \columnratio{0.35}
    \setlength{\columnsep}{7mm}
    \begin{paracol}{2}

    \sectiontitle{links}
    \sectioncontent{
        \faIcon{github-alt}\hspace{2mm}
        \href{https://github.com/keagud}{github.com/keagud} \\
      %  \faIcon{linkedin-in}\hspace{2.1mm}
      %  \href{example.com}{linkedin} \\
        \faIcon{link}\hspace{1.8mm}
        \href{keagud.github.io}{keagud.github.io}
    }

    \sectiontitle{education}
    \sectioncontent{
        \begingroup
      \textbf{University of Washington, Seattle }\hfill\color{black!70}\small{2016-2020}
        \endgroup
      \begin{itemize}
      \item Overall GPA: 3.88, graduated \textit{Cum Laude}
      \item Made the Dean's List for every term that met the credit requirement (all but 2)
      \item Phi Beta Kappa member since 2018
        \end{itemize}
      \item \textbf{Bachelor of Arts, Linguistics }
        \begin{itemize}
          \item Relevent Courses: 
            \begin{itemize}
              \item Syntax I-II: Theory and analysis of both natural and formal language systems.
              \item Intro to Computational Linguistics: Survey of natural language processing methods including deterministic, Baysian, and machine learning paradigms, with accompanying practical experience with Python.
            \end{itemize}

        \end{itemize}
      \item \textbf{Bachelor of Science, Speech \& Hearing Science}
      } 

    \sectiontitle{Hobbies \& Interests}
    \sectioncontent{
      \begin{itemize}
        \item Ancient languages \\ (Latin, Attic Greek)
        \item Ancient languages \\  (Lisp, C89, Ada)
        \item \href{https://en.wikipedia.org/wiki/Constructed\_language}{Conlanging}
        \item Distance running, cycling

      \end{itemize}
    }

    \switchcolumn
    

    \sectiontitle{Technical Skills}
    \sectioncontent{
      \begin{itemize}
        \item \textbf{Programming Languages: }
          \begin{itemize}
            \item \textit{Proficient:} Python, C, C++, Javascript/Typescript
            \item \textit{Actively Learning:} Rust, Java
          \end{itemize}

        \item \textbf{Other Technical Skills:}
          \begin{itemize}
            \item  Linux user for all personal computing needs for the past 7 years, adept in most common command line utilities in the Linux/Unix environment (bash/zsh, sed, grep, ssh, vim...), as well as the Unix filesystem architecture and API
            \item Experienced with the use of version control with Git
          \end{itemize}
      \end{itemize}
    }

        \sectiontitle{Experience} 
        \sectioncontent{
        \begingroup
      \textbf{Technical Consultant - Ikolith Recreations}\hfill\color{black!70}\small{2020-2022}
        \endgroup
      \begin{itemize}
          \item Technical and creative work with a local Seattle-based independent tabletop game studio. 
              \begin{itemize}
                  \item Wrote and managed utility scripts, mostly in Python and bash, to automate workflow tasks such as document formatting.
                    \item Developed an automated pipeline to convert and compile Markdown documentation into Latex for publication.
                    \item Created a program to partially simulate the evolution of a fictional language over time as a worldbuilding and writing aid.
                        
                        
              \end{itemize}
        \end{itemize}
        }


        
    \sectiontitle{projects}
    \sectioncontent{
      %%%%%%%%%%%%%%%%%%%%%%%%%
      \project{\href{https://github.com/keagud/della}{Della}}{Python}
        \begin{itemize}
          \item A personal orginizer similar to TaskWarrior, with the added feature of natural language processing to allow todos to be managed with a simple, naturalistic command syntax, e.g. "pick up Dave at the airport a week from this Wednesday." 
          \item Users can interact with and view their upcoming tasks either from the command line, or from a formatted prompt featuring full syntax highlighting. Tasks can be set to sync to a server for access from multiple devices.
          \item Written in pure \textbf{Python}; remote syncing is handled over \textbf{SSH} via the \textbf{Paramiko} library.
        \end{itemize}
        \spacevv
        %%%%%%%%%%%%%%%%%%%%%%%
        
        %%%%%%%%%%%%%%%%%%%%%%%%%
        \project{\href{https://github.com/keagud/Quackbot}{Quackbot}}{Javascript, Typescript}
        \begin{itemize}
            \item A Discord bot with a modular design, allowing features and commands to be easily swapped out. Current functionality includes: fortune telling with a simulated Tarot deck, generation of images based on a macro, user search of Wikipedia. 
            \item Written in \textbf{Typescript} and \textbf{Javascript} using \textbf{Node.js} and \textbf{npm}. Hosted on \textbf{Google Cloud} with building and deployment fully automated using a \textbf{Github Actions CI/CD } workflow script.
        \end{itemize}
        \spacevv
        %%%%%%%%%%%%%%5

        
        %%%%%%%%%%%%%%%%%%%%%%%%%%%%%%
        \iffalse
        \project{\href{https://github.com/keagud/dateparse}{Dateparse}}{Python}
        \begin{itemize}
          \item A pure Python, dependency-free library for parsing natural language expressions about time, like "two weeks ago", "a week before Christmas", "in four months" into Python date objects. 
        \end{itemize}
        \spacevv
        \fi

        %%%%%%%%%%%%%%%%%%%%%%%%%%%%%
        \project{\href{https://github.com/keagud/linelog}{Linelog}}{Python}
        \begin{itemize}
          \item A development productivity tool that calculates the number of lines of code commited in Git by a given user, over a given interval of time, and displays the results as a curses-style graph on the command line broken down by language. Definitions for what should count as a "line" are fully customizable for each filetype.
        \end{itemize}
        \spacevv
        %%%%%%%%%%%%%%%%%%

      \iffalse
        \project{\href{https://github.com/keagud/verba}{Verba Whitakari}}{C++, Python, CMake}
        \begin{itemize}
          \item A complete port of the classic Latin morphological analysis and dictionary tool \href{https://en.wikipedia.org/wiki/William\_Whitaker\%27s\_Words>}{William Whitaker's Words}, translated and refactored from the original 1993 Ada program into modern C++. This project aims to maintain the functionality of the original in a form that can be better preserved into the future (Ada build systems are not trivial to work with in \$CURRENT\_YEAR) while improving performance and ease of use. 
        \end{itemize}    
        \fi
        %%%%%%%%%%%%%%%%%%%%%%%%%%%%
}
    \end{paracol}

\end{document}
