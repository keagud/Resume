

\documentclass[11pt]{article}


% ================ PACKAGES ====================

\usepackage[reset, a4paper, margin=6mm, top=7mm, right=4mm]{geometry}
% text alignment
\usepackage{ragged2e}
% loremipsum
\usepackage{lipsum}
% text color
\usepackage{color}
\usepackage{xcolor}
% underline
\usepackage{ulem}
\usepackage{contour} 
% multiple columns
\usepackage{paracol}
% custom fonts
\usepackage{fontspec}
% paper bg
\usepackage{eso-pic}
% icons
\usepackage{fontawesome5}
% hyperlinks
\usepackage{hyperref}
% better itemize
\usepackage{enumitem}

\usepackage[abspath]{currfile}

% ================ SETTINGS ====================

\setlist[itemize]{noitemsep, topsep=-2pt, leftmargin=2ex}

\setmainfont{IBMPlexSerif} [
  Extension = .otf , 
  Path = fonts/ ,
  UprightFont = *-Regular ,
  ItalicFont = *-Italic , 
  BoldFont = *-Bold ,
  BoldItalicFont = *-BoldItalic ]




\setsansfont{IBMPlexSans} [
  Extension = .otf , 
  Path = fonts/ ,
  UprightFont = *-Regular ,
  ItalicFont = *-Italic , 
  BoldFont = *-Bold ,
  BoldItalicFont = *-BoldItalic ]


\renewcommand{\familydefault}{\rmdefault}

% no page numbers and the sorts
\thispagestyle{empty}
% no indentation at beginning of paragraph
\parindent=0pt
% for underline
\renewcommand{\ULdepth}{3pt}
\renewcommand{\ULthickness}{0.5pt}
\contourlength{2.0pt}

% for links
\hypersetup{
  urlbordercolor=gray,
 pdfborderstyle={/S/U/W 1}
}

% no hyphenation
\tolerance=1
\emergencystretch=\maxdimen
\hyphenpenalty=10000
\hbadness=10000

% ================ MACROS ======================

\newfontfamily\titlethin[Path=fonts/]{IBMPlexSerif-Regular}
\newfontfamily\titlethick[Path=fonts/]{IBMPlexSerif-SemiBold}
\newfontfamily\titlebold[Path=fonts/]{IBMPlexSerif-Bold}

\newfontfamily\normaltext[Path=fonts/]{IBMPlexSerif-Regular}

\newcommand{\resumetitle}[3]{
    \AddToShipoutPictureBG{
        \AtPageUpperLeft {
        \raisebox{-0.09\paperheight}{
            \color{black!85}\rule{2\paperwidth}{\paperheight}}
        }}
    \begin{Center}
        \begingroup
        \titlethin
        \color{black!10}\Huge{#1}
        \titlethick
        \color{black!5}\Huge{#2} \\
        \vspace{2mm}
        \textrm{\color{black!15}\Large{#3}}
        \endgroup
    \end{Center}
    \vspace{7mm}
}

\newcommand{\betteruline}[1]{
    \uline{#1}
}

\newcommand{\sectiontitle}[1]{
    \begingroup
        \titlebold
        \betteruline{\Large\uppercase{#1}  }
        \vspace{1.7mm}
    \endgroup
}

\newcommand{\sectioncontent}[1]{
    \begingroup
        \begin{FlushLeft}
        \vspace{-3mm}
        \sffamily\small#1
        \end{FlushLeft}
    \endgroup
    \vspace{2mm}
}

\newcommand{\job}[3]{
    \begingroup
        \textbf{\small#1} - \small#2
        \hfill\color{black!70}\small{#3}
    \endgroup
}

\newcommand{\project}[2]{
    \begingroup
        \textbf{\small#1}
        \hfill\color{black!70}\small{#2}
    \endgroup
}

\newcommand{\spacevv}{
    \vspace{2mm}
}

\newcommand{\honor}[2]{
    \textcolor{black!70}{#1} - #2 \\
    \vspace{1.5mm}
}

\begin{document}
    \resumetitle{Keaton}{Guderian} {
      keagud@protonmail.com | (206)-321-2181
    }

    \columnratio{0.35}
    \setlength{\columnsep}{7mm}
    \begin{paracol}{2}

    \sectiontitle{links}
    \sectioncontent{
        \faIcon{github-alt}\hspace{2mm}
        \href{{{https://github.com/keagud}}}{github.com/keagud} \\
      %  \faIcon{linkedin-in}\hspace{2.1mm}
      %  \href{example.com}{linkedin} \\
        \faIcon{link}\hspace{1.8mm}
        \href{keagud.github.io}{keagud.github.io}
    }

    \sectiontitle{education}
    \sectioncontent{
        \begingroup
      \textbf{University of Washington, Seattle }\hfill\color{black!70}\small{2016-2020}
        \endgroup
      \begin{itemize}
      \item Overall GPA: 3.88, graduated \textit{Cum Laude}
      \item Phi Beta Kappa member since 2018
        \end{itemize}
      \item \textbf{Bachelor of Arts, Linguistics }
        \item \textbf{Bachelor of Science, Speech \& Hearing Science}
      } 

    \sectiontitle{Hobbies \& Interests}
    \sectioncontent{
      \begin{itemize}
        \item Ancient languages \\ (Latin, Attic Greek)
        \item Ancient languages \\  (Lisp, C89, Ada)
        \item \href{https://en.wikipedia.org/wiki/Constructed\_language}{Conlanging}
        \item Distance running, cycling

      \end{itemize}
    }

    \switchcolumn
    
      

    \sectiontitle{Technical Skills}
    \sectioncontent{
      
      
      \begin{itemize}
        \item \textbf{Programming Languages: }
          \begin{itemize}
          
          \item For 10 years:  Python
          
          \item For 7 years:  HTML, Javascript
          
          \item For 3 years:  Rust, C, Typescript
          
          \end{itemize}


        
        

        \item \textbf { Technologies and Frameworks }
        
          \begin{itemize}
            \item Django, Flask, SQL, Docker, Svelte, React
          
            \smallskip
          
          \item Linux user for all personal computing needs for the past 7 years, adept in most common command line utilities in the Linux/Unix environment (bash/zsh, sed, grep, ssh, vim...), as well as the Unix filesystem architecture and API
          
          
          \end{itemize}
        
        

      \end{itemize}


    }

        \sectiontitle{Experience} 
        \sectioncontent{

    
    

        
        
        

        
          
        


        \begingroup
      \textbf{ Ikolith Recreations }\hfill\color{black!70}\small{ 2021 - 2022 }
        \endgroup
        \textit{ Contractor }

        
      \begin{itemize}

        
        \item Contract work doing technical "odd jobs" for a local Seattle-based independent tabletop game studio.
            
            
              
                  \item Wrote and managed utility scripts, mostly in Python and bash, to automate workflow tasks such as document formatting.
                
                  \item Developed an automated pipeline to convert and compile Markdown documentation into Latex for publication.
                
                  \item Created a program to partially simulate the evolution of a fictional language over time as a worldbuilding and writing aid
                
            
      \end{itemize}
      
        
    

        
        
        

        
          
        


        \begingroup
      \textbf{ Boxwave }\hfill\color{black!70}\small{ July 2023 - October 2023 }
        \endgroup
        \textit{ Programming Intern }

        
      \begin{itemize}

        
            
              
                  \item Wrote tools in Python and C Sharp to automatically fetch up-to-date vendor reports via Amazon's seller API, providing an overall efficiency increase of over 100%
                
                  \item Designed, implemented, and tested database schemas representing complex product relationships for use with Postgres
                
                  \item Trained and deployed a Recurrent Neural Network using Tensorflow to automate various orginization tasks involving optical character recognition
                
            
      \end{itemize}
      
        
        }




        
    \sectiontitle{projects}
    \sectioncontent{
      %%%%%%%%%%%%%%%%%%%%%%%%%

    
      
        
          \project{ \href{https://implicit.computer}{Personal Website} }{ Rust,HTML/CSS,Javascript  }
        \begin{itemize}

            
              

              \item My personal website, which contains my blog as well as some interactive experiments in client-side development.

                

              \item The backend, including template rendering, routing, and database integrations, was made in Rust using the Axum library.

                
            

        \end{itemize}

        
        \spacevv
        

    
        
          \project{ \href{https://github.com/keagud/linelog}{Linelog} }{ Python  }
        \begin{itemize}

            
              

              \item A development productivity tool that calculates the number of lines of code commited in Git by a given user, over a given interval of time, and displays the results as a curses-style graph on the command line broken down by language. Definitions for what should count as a "line" are fully customizable for each filetype.

                
            

        \end{itemize}

        
        \spacevv
        

    
        
          \project{ \href{https://github.com/keagud/quackbot}{Quackbot} }{ Javascript,Typescript,NodeJS  }
        \begin{itemize}

            
              

              \item A Discord bot with a modular design, allowing features and commands to be easily swapped out. Current functionality includes: fortune telling with a simulated Tarot deck, generation of images based on a macro, user search of Wikipedia.

                

              \item Written in Typescript and Javascript using Node.js and npm. Hosted on Google Cloud, with building and deployment fully automated using a Github Actions CI/CD workflow script

                
            

        \end{itemize}

        

    
        %%%%%%%%%%%%%%%%%%%%%%%
}
    \end{paracol}

\end{document}